%%%%%%%%%%%%%%%%%%%%%%%%%%%%%%%%%%%%%%%%%
%  My documentation report
%  Objetive: Explain what I did and how, so someone can continue with the investigation
%
% Important note:
% Chapter heading images should have a 2:1 width:height ratio,
% e.g. 920px width and 460px height.
%
%%%%%%%%%%%%%%%%%%%%%%%%%%%%%%%%%%%%%%%%%


%----------------------------------------------------------------------------------------
%	PACKAGES AND OTHER DOCUMENT CONFIGURATIONS
%----------------------------------------------------------------------------------------

\documentclass[11pt,fleqn]{book} % Default font size and left-justified equations

\usepackage[top=3cm,bottom=3cm,left=3.2cm,right=3.2cm,headsep=10pt,letterpaper]{geometry} % Page margins

\usepackage{xcolor} % Required for specifying colors by name
\definecolor{ocre}{RGB}{52,177,201} % Define the orange color used for highlighting throughout the book

% Font Settings
\usepackage{avant} % Use the Avantgarde font for headings
%\usepackage{times} % Use the Times font for headings
\usepackage{mathptmx} % Use the Adobe Times Roman as the default text font together with math symbols from the Sym­bol, Chancery and Com­puter Modern fonts
\usepackage{microtype} % Slightly tweak font spacing for aesthetics
\usepackage[utf8]{inputenc} % Required for including letters with accents
\usepackage[T1]{fontenc} % Use 8-bit encoding that has 256 glyphs
\usepackage{amsthm}
\usepackage{hyperref}


\begin{document}

%----------------------------------------------------------------------------------------
%	TITLE PAGE
%----------------------------------------------------------------------------------------

\tableofcontents % Print the table of contents itself

\chapter{WSL Install}\label{chpter:chapter 1}
This applies only if you are making use of windows 10 or 11. 
You can decide to skip this chapter a s a docker alternative for running 
ROS is being prepared but you might have compatibility issues, thus I adice you go through this 
painfully.
\section{powershell install}
\begin{itemize}
    \item Search for powershell
    \item Right click on it and click run as administrator
    \item Enter 'wsl --list -online'
    \item In the results, check for latest version having LTS attached whichmeans long time support
    \item Enter 'wsl --install -d <DISTRO-NAME>'. In this case <DISTRO-NAME> is ubuntu-20.04.
    \item Restart  your computer to finish the installation
    \item Search for ubuntu and you can drag it to taskbar icon for easy startup.
\end{itemize}
\section{microsoft store install}
\begin{itemize}
    \item Open start
    \item Search for 'Turn windows features on or off'
    \item On the resulting window, scroll down and search for 'Virtual Machine Platform'
    \item Ensure the 'Virtual Machine Platform' is ticked then click 'OK'
    \item Retart your system
    \item After system restart, open the microsoft store app
    \item search for ubuntu
    \item selct the latest version with LTS support
    \item Then install.
    \item After install, search for ubuntu on system and you can drag it to taskbar icon for easy startup.
\end{itemize}

\chapter{Docker Install}
\section{On Windows}
I would advice you go through \ref{chpter:chapter 1} now and install WSL appropriately. This will ensusre you ease of setup of 
otherpackages in the future.
\begin{itemize}
    \item Go to \href{https://docs.docker.com/desktop/install/windows-install/}{Docker download page}
    \item Click on 'Download Desktop for Windows'
    \item Run the downloaded installer
    \item Select WSL2 backend if you went through with the first chapter, else select hyper-V backend option.
    \item Follow the remaining instructions to install and click close when installation is complete.
    \item If your user accoount is not the admin account,search for 'Computer Management' and right click to run as administrator.
    \item Go to 'Local Users and Groups'
    \item Under 'Local Users and 'Groups' go to 'Groups'
    \item Under 'Groups', select 'docker-users'
    \item Right click to add the user to the group.
    \item Log out and log back in so chnages can be applied 
\end{itemize}
\section{On Ubuntu, other linux distros}
\begin{itemize}
    \item Open a new terminal using Ctrl+Alt+T
    \item Enter 'sudo apt-get remove docker docker-engine docker.io containerd runc'
    \item It is okay if you get a result that none of the packages are installed.
    \item Then enter 'sudo apt-get update'
    \item Then Enter 
        'sudo apt-get install \
            ca-certificates \
            curl \
            gnupg \
            lsb-release'
    \item Enter 'sudo mkdir -m 0755 -p /etc/apt/keyrings'
    \item Enter 'curl -fsSL https://download.docker.com/linux/ubuntu/gpg | sudo gpg --dearmor -o /etc/apt/keyrings/docker.gpg'
    \item Enter 
        'echo \
        "deb [arch=\$(dpkg --print-architecture) signed-by=/etc/apt/keyrings/docker.gpg] https://download.docker.com/linux/ubuntu \
        \$(lsb\_release -cs) stable" | sudo tee /etc/apt/sources.list.d/docker.list > /dev/null'
    \item Enter 'sudo chmod a+r /etc/apt/keyrings/docker.gpg'
    \item Enter 'sudo apt-get update'
    \item Enter 'sudo apt-get install docker-ce docker-ce-cli containerd.io docker-buildx-plugin docker-compose-plugin'
    \item Enter 'sudo docker run hello-world'
\end{itemize}
The last command should print a message and then exit. 

\chapter{Setup ROS on Docker}
This part will be updated soon

\chapter{Useful Resources}
\section{Installations}
\begin{itemize}
    \item \hyperlink{https://github.com/emmanuel-olateju/ROS_tutorial/blob/main/00.essentials_setup/00.Essentials_setup.pdf}{IEEE-OAUSB-RAS ROS classes setups}
\end{itemize}
\section{Linux}
\begin{itemize}
    \item \hyperlink{https://www.youtube.com/watch?v=gd7BXuUQ91w}{60 linux commands}
    \item \hyperlink{https://www.nano-editor.org/dist/latest/cheatsheet.html}{nano editor cheatsheet}
\end{itemize}
\section{ROS}
\begin{itemize}
    \item \hyperlink{https://github.com/emmanuel-olateju/ROS_tutorial}{petron's ROS github}
    \item \hyperlink{https://github.com/micro-ROS/}{micro-ROS github}
\end{itemize}


\end{document}